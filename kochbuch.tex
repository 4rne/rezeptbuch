\documentclass[%
a4paper,
%twoside,
11pt
]{article}

% encoding, font, language
\usepackage[T1]{fontenc}
\usepackage[utf8]{inputenc}
\usepackage{lmodern}
\usepackage[ngerman]{babel}

\usepackage{amsmath}
\numberwithin{page}{section}
\renewcommand{\thesection}{\Roman{section}}
\renewcommand{\thepage}{\thesection~ - \arabic{page}}

\usepackage{wasysym}

%\usepackage{mathptmx}
\usepackage[sfdefault,light]{roboto}
\usepackage{lipsum}

\usepackage{nicefrac}

\usepackage[
%    handwritten,
    nowarnings,
    %myconfig
]
{xcookybooky}

\usepackage{blindtext}    % only needed for generating test text

\DeclareRobustCommand{\textcelcius}{\ensuremath{^\circ}C}


\setcounter{secnumdepth}{1}
\renewcommand*{\recipesection}[2][]
{%
    \subsection[#1]{#2}
}
\renewcommand{\subsectionmark}[1]
{% no implementation to display the section name instead
}


\title{Rezeptsammlung}
\author{Arne}

\usepackage{hyperref}    % must be the last package
\hypersetup{%
    pdfauthor            = {Arne},
    pdftitle             = {Rezeptsammlung},
    pdfsubject           = {Recipes},
    pdfkeywords          = {example, recipes, cookbook, xcookybooky},
    pdfstartview         = {FitV},
    pdfview              = {FitH},
    pdfpagemode          = {UseNone}, % Options; UseNone, UseOutlines
    bookmarksopen        = {true},
    pdfpagetransition    = {Glitter},
    colorlinks           = {true},
    linkcolor            = {black},
    urlcolor             = {blue},
    citecolor            = {black},
    filecolor            = {black},}

\hbadness=10000	% Ignore underfull boxes

\begin{document}

%\maketitle

%\begin{abstract}
%    \noindent The examples in this document require at least version~1.4 of the \texttt{xcookybooky}\footnote{\url{http://www.ctan.org/pkg/xcookybooky}} package. For more examples and test recipes especially for using hook functions take a look at the source files located at \url{https://code.google.com/p/xcookybooky/}. If you are interested in modifying the layout of \texttt{xcookybooky} you will find examples in the documentation as well as in the configuration file \textbf{xcookybooky.cfg}.
%\end{abstract}

\tableofcontents

\vspace{5em}

% ===========================================================================================================================

\section{Kochen}
\setcounter{page}{1}
\pagebreak

% background graphic
%\setBackgroundPicture[x, y=-2cm, width=\paperwidth-4cm, height, orientation = pagecenter]
%{pic/background}

\setHeadlines
{% translation
    inghead = Zutaten,
    prephead = Zubereitung,
    hinthead = Tipp,
    continuationhead = Fortsetzung,
    continuationfoot = Fortsetzung auf n\"achster Seite,
    portionvalue = Personen,
}

\begin{recipe}
[ % Optionale Eingaben
    preparationtime = {\unit[1]{h}},
    portion = \portion{4},
    bakingtime={\unit[30-45]{min}},
    bakingtemperature={\unit[200-250]{\textcelcius}},
    source = Antje
]
{Kartoffel-Lachs-Auflauf}
    
    %\graph
    %{% Bilder
    %    small=pic/glass,    % kleines Bild
    %    big=pic/ingredients % großes (längeres) Bild
    %}
    
    \ingredients
    {% Zutaten
        \unit[1,5]{kg} & Kartoffeln \\
	\unit[200]{g} & Lachs \\
        \unit[200]{ml} & Sahne \\
        \unit[200]{ml} & Milch \\
        2 & Eier \\
        1 & Bund Dill \\
	& Pfeffer, Salz \\
        1 & Zitrone (optional)\\
    }
    
    \preparation
    { % Zubereitung
	\step Backofen vorheizen
        \step Pellkartoffeln kochen, pellen, in Scheiben schneiden. Lachs in Streifen schneiden.
        \step Kartoffeln und Lachs dachziegelartig in eine gefettete Auflaufform legen.
        \step Soße aus Sahne, Milch, Eiern, Dill, Salz und Pfeffer darübergeben.
        \step Eine Dreiviertelstunde bei 200°-250°C backen.
    }
    
        \suggestion[Variante: Nudel-Lachs-Auflauf]
    {%
        Statt Kartoffeln 500g gekochte Nudeln verwenden und eine halbe Stunde backen.
    }
    
    \hint
    {% Tipp
      Mit Salat servieren
    }

\end{recipe}

\pagebreak

\begin{recipe}
[% 
    preparationtime = {\unit[1,5]{h}},
    bakingtime={\unit[35-45]{min}},
    bakingtemperature={\unit[210-230]{\textcelcius}},
    portion = {\portion{4}},
    %calory={\unit[3]{kJ}},
 %   source = {Handbuch für das tägliche Backen}
]
{Quiche lorraine}
    
    %\graph
    %{% pictures
    %    small=pic/glass,     % small picture
    %    big=pic/ingredients  % big picture
    %}
    
    \ingredients{%
        \unit[200]{g} & Mehl \\
        \unit[150]{g} & Butter \\
        \unit[\nicefrac{1}{2}]{TL} & Salz \\
        \unit[3-4]{EL} & Wasser \\
        \multicolumn{2}{l}{\textbf{Belag:}} \\
        \unit[150]{g} & gekochter Schinken \\
        \unit[150]{g} & Emmentaler oder Gouda \\
        4 & Eier \\
        \unit[125]{ml} & saure Sahne \\
        \unit[\nicefrac{1}{2}]{TL} & Paprikapulver \\
        & etwas Petersilie \\
    }
    
    \preparation{%
	\step Backofen vorheizen.
        \step Das Mehl in eine Schüssel geben, das Wasser mit etwas Mehl verrühren. 
        Salz zufügen und die Buttern in Flöckchen darauf verteilen. 
        Alles zu einem glatten Teig verkneten und 1 Stunde ruhen lassen. 
        \step Den Teig ausrollen und in eine gefettete Springform (\unit[24-26]{cm} \diameter) geben; einen \unit[3]{cm} hohen Rand andrücken. 
        Die Sahne mit den Eiern verquirlen und den feingewürfelten Schinken und Käse mit Paprika und gehackter Petersilie unterrühren. 
        Die Masse auf den Teigboden verteilen.
        \step 35-45 Minuten backen.
    }
    
%    \hint{%
%        Enjoy typesetting recipes with {\textbf{\Large\LaTeX}} and {\textbf{\Large xcookybooky!}}
%    }
    
\end{recipe}

\pagebreak

\begin{recipe}
[% 
    preparationtime = {\unit[30]{min}},
    portion = {\portion{4}},
    %calory={\unit[3]{kJ}},
    source = Heilke
]
  {Spinatsuppe}

  %\graph
  %{% pictures
  %    small=pic/glass,     % small picture
  %    big=pic/ingredients  % big picture
  %}

  \ingredients{%
    \unit[1]{kg} & Tiefkühl-Spinat \\
    \unit[200]{g} & Sahne-Schmelzkäse \\
    1 & Zwiebel, gewürfelt \\
    & Weizenmehl \\
    & Margarine \\
    & Milch \\
    & Sahne \\
    & Salz, Pfeffer, Muskat
  }

  \preparation{%
    \step Spinat im Topf erhitzen.
    \step In einem weiteren Topf das Fett zerlassen und die Zwiebel darin bräunen. Mit etwas Mehl eine Mehlschwitze  bereiten und mit etwas Milch auffüllen.
    \step Den erhitzten Spinat und den Käse dazugeben und mit Gewürzen abschmecken.
    \step Mehrmals umrühren und aufkochen lassen. Zum Schluss einen Schuss Sahne zugeben.
  }
  
\end{recipe}

% Complete recipe example
\begin{recipe}
[% 
    preparationtime = {\unit[1]{h}},
    bakingtime={\unit[1]{h}},
    portion = {\portion{4}},
    %calory={\unit[3]{kJ}},
    source = {Opa Norderstedt, Rezept aus Lage}
]
{Pickert}
    
    %\graph
    %{% pictures
    %    small=pic/glass,     % small picture
    %    big=pic/ingredients  % big picture
    %}
    \ingredients{%
        \unit[1]{kg} & geriebene Kartoffeln \\
        \unit[1]{kg} & Mehl \\
        \unit[200]{g} & Hefe \\
        \unit[125]{g} & Rosinen \\
        \unit[125-250]{ml} & Milch \\
        3 - 4 & Eier \\
        \unit[2]{Stk.} & Palmin \\
        & etwas Salz und Zucker \\
    }
    
    \preparation{%
      \step Hefestück mit warmer Milch und Zucker ansetzen.
      \step Inzwischen alles vermengen (möglichst alle Zutaten gut vorwärmen), Hefe dazugeben. Es muss ein dicklicher Teig sein. Diesen \unit[2]{Stunden} an warmer Stelle gehen lassen.
      \step Die Pfannkuchen (fingerdick) in Backöl oder Palmin langsam in der Pfanne backen.
    }
    
\end{recipe}

% ============================================================================================================================
\pagebreak
\section{Backen}
\setcounter{page}{1}

\begin{recipe}
[% 
    preparationtime = {\unit[15]{min}},
    bakingtime={\unit[60-70]{min}},
    bakingtemperature={\unit[220]{°C}},
    %calory={\unit[3]{kJ}},
    source = Kathrin
]
{Kathrins Brot}
    
    %\graph
    %{% pictures
    %    small=pic/glass,     % small picture
    %    big=pic/ingredients  % big picture
    %}
    
    \ingredients{%
        \unit[400]{g} & Weizenmehl (\nicefrac{1}{2} Vollkorn) \\
        \unit[225]{g} & Roggenmehl \\
        \unit[500]{ml} & lauwarmes Wasser \\
        1 & Trockenhefe \\
        \unit[2]{TL} & Salz \\
        \unit[1]{TL} & Zucker \\
        & Kerne nach Wahl
    }
    
    \preparation{%
      \step Die Zutaten verkneten.
      \step Den Teig in eine gefettete Kastenform geben.
      \step Teig \unit[1]{Stunde} warm gehen lassen.
      \step Bei \unit[220]{°C} 60 - 70 Minuten backen. Eventuell mit Alufolie abdecken.
    }
    
\end{recipe}

\pagebreak
\begin{recipe}
[% 
    preparationtime = {\unit[60]{min}},
    bakingtime={\unit[15]{min}},
    bakingtemperature={\unit[200]{°C}},
    source = {Angepasstes Rezept nach \url{www.hamburg.de/franzbroetchen}}
]
{Hamburger Franzbrötchen}
    
    %\graph
    %{% pictures
    %    small=pic/glass,     % small picture
    %    big=pic/ingredients  % big picture
    %}
    
    \ingredients{%
        \unit[500]{g} & Mehl \\
        \unit[70]{g} & Zucker \\
        \unit[70]{g} & weiche Butter \\
        \unit[250]{ml} & lauwarme Milch \\
        & Würfel Hefe \\
	& Prise Salz \\
        \multicolumn{2}{l}{\textbf{Füllung:}} \\
        \unit[200]{g} & kalte Butter \\
        \unit[4-5]{TL} & Zimt \\
        \unit[250-300]{g} & Zucker \\
    }
    
    \preparation{%
	\step Aus den Zutaten für den Teig einen Hefeteig herstellen, gut durchkneten und auf einer bemehlten Arbeitsfläche zu einem Rechteck ausrollen (ca. 30 x 25 cm). 
	
	\step \unit[100]{g} der kalten Butter in dünne Scheiben schneiden und auf eine Teighälfte legen, die andere Teighälfte darüber klappen und dabei achtgeben, dass sich keine Blasen unter dem Teig bilden. 
	Dann die Teigränder zusammendrücken, Ränder unter den Teig schieben. Den Teig auf ca. 50 x 30 cm ausrollen. 
	Ein Drittel des Teiges von links zur Mitte hin einschlagen, das restliche Drittel von rechts darüber schlagen, sodass drei Schichten Teig übereinander liegen. 
	Auch hier ist darauf zu achten, dass keine Luftblasen im Teig entstehen.
	
	\step Den Teig 20 Minuten kühl stellen. währenddessen die Füllung Zimt und Zucker mit der inzwischen in einem Topf geschmolzenen Butter verrühren. 
	
	\step Den kalten Teig auf ca. 80 x 40 cm ausrollen. Die Füllung bis auf zwei Esslöffel gleichmäßig auf den unteren zwei Dritteln des Teiges verteilen. 
	Von unten her vorsichtig den Teig von der Arbeitsfläche lösen und aufrollen, leicht andrücken. 
	\step Die Rolle in ca. 4 cm breite Stücke schneiden und die Franzbrötchen in einem Abstand von mindestens \unit[10]{cm} auf ein Backblech legen.
	Dort mit dem Stiel eines Kochlöffels in der Mitte fast ganz durchdrücken, sodass der Teig sich über dem Stiel zusammenwölbt und die Füllung und der Teig aus der Mitter herausquillt.
	Mit der übrigen Füllung nun die Franzbrötchen bestreichen.
	
	\step Den Backofen auf 200°C vorheizen und die Franzbrötchen abgedeckt mindestens 20 Minuten gehen lassen.
	\step Im vorgeheizten Backofen bei Umluft ca. 15 Minuten backen, bis die typische Bräune erreicht ist.
    }
   
%    \hint{%
%        Warm genießen.
%    }
    
\end{recipe}

\begin{recipe}
[% 
    preparationtime = {\unit[4]{h}},
    bakingtime={\unit[15-17]{min}},
    bakingtemperature={\unit[150]{\textcelcius}},
    portion = {\portion{4}},
    %calory={\unit[3]{kJ}},
    source = {Opa Norderstedt, Rezept aus Lage}
]
{Heidesand}
    
    %\graph
    %{% pictures
    %    small=pic/glass,     % small picture
    %    big=pic/ingredients  % big picture
    %}
    \ingredients{%
        \unit[250]{g} & Butter \\
        \unit[250]{g} & Zucker \\
        \unit[375]{g} & Mehl \\
        1 & Vanillezucker \\
        & Prise Salz \\
    }
    
    \preparation{%
      \step Butter bräunen und abkühlen. Mit Zucker, Salz und Vanillezucker schaumig rühren und \unit[4]{cm} dicke Rollen formen und in Backpapier gerollt einige Stunden kühl legen.
      \step Von den Rollen \unit[\nicefrac{1}{2}]{cm} dicke Scheiben abschneiden und auf das gefettete Blech legen. 
    }
    
\end{recipe}

\begin{recipe}
[% 
    preparationtime = {\unit[\nicefrac{3}{4}]{h}},
    portion = {\portion{4}},
    %calory={\unit[3]{kJ}},
    source = {Opa Norderstedt, Rezept aus Lage}
]
{Rumtrüffel (Konfekt)}
    
    %\graph
    %{% pictures
    %    small=pic/glass,     % small picture
    %    big=pic/ingredients  % big picture
    %}
    \ingredients{%
        \unit[125]{g} & Zucker \\
        \unit[125]{g} & Kokosfett \\
        \unit[70]{g} & gemahlene Mandeln \\
        3 & gehäufte EL Kakao \\
        2 & Eier \\
        & Schokoladenstreusel \\
        & Schuss Rum oder Rumaroma \\
    }
    
    \preparation{%
      \step Eier mit Zucker schaumig rühren, Kakao, Kokosfett, Mandeln, Rum(-aroma) dazugeben.
      \step Kalt stellen.
      \step Mit einem Löffel Kugeln formen und in Schokoladenstreuseln wälzen.
    }
    
\end{recipe}

\end{document} 