\documentclass[%
a4paper,
%twoside,
11pt
]{article}

% encoding, font, language
\usepackage[T1]{fontenc}
\usepackage[utf8]{inputenc}
\usepackage{lmodern}
\usepackage[ngerman]{babel}

\usepackage{mathptmx}

\usepackage{lipsum}

\usepackage{nicefrac}

\usepackage[
%    handwritten,
    nowarnings,
    %myconfig
]
{xcookybooky}

\usepackage{blindtext}    % only needed for generating test text

\DeclareRobustCommand{\textcelcius}{\ensuremath{^{\circ}\mathrm{C}}}


\setcounter{secnumdepth}{1}
\renewcommand*{\recipesection}[2][]
{%
    \subsection[#1]{#2}
}
\renewcommand{\subsectionmark}[1]
{% no implementation to display the section name instead
}


\title{Rezeptsammlung}
\author{Arne}

\usepackage{hyperref}    % must be the last package
\hypersetup{%
    pdfauthor            = {Arne},
    pdftitle             = {Rezeptsammlung},
    pdfsubject           = {Recipes},
    pdfkeywords          = {example, recipes, cookbook, xcookybooky},
    pdfstartview         = {FitV},
    pdfview              = {FitH},
    pdfpagemode          = {UseNone}, % Options; UseNone, UseOutlines
    bookmarksopen        = {true},
    pdfpagetransition    = {Glitter},
    colorlinks           = {true},
    linkcolor            = {black},
    urlcolor             = {blue},
    citecolor            = {black},
    filecolor            = {black},}

\hbadness=10000	% Ignore underfull boxes

\begin{document}

%\maketitle

%\begin{abstract}
%    \noindent The examples in this document require at least version~1.4 of the \texttt{xcookybooky}\footnote{\url{http://www.ctan.org/pkg/xcookybooky}} package. For more examples and test recipes especially for using hook functions take a look at the source files located at \url{https://code.google.com/p/xcookybooky/}. If you are interested in modifying the layout of \texttt{xcookybooky} you will find examples in the documentation as well as in the configuration file \textbf{xcookybooky.cfg}.
%\end{abstract}

\tableofcontents

\vspace{5em}

\section{Rezepte}
\pagebreak

% background graphic
%\setBackgroundPicture[x, y=-2cm, width=\paperwidth-4cm, height, orientation = pagecenter]
%{pic/background}

\setHeadlines
{% translation
    inghead = Zutaten,
    prephead = Zubereitung,
    hinthead = Tipp,
    continuationhead = Fortsetzung,
    continuationfoot = Fortsetzung auf n\"achster Seite,
    portionvalue = Personen,
}

\begin{recipe}
[ % Optionale Eingaben
    preparationtime = {\unit[1]{h}},
    portion = \portion{5},
    source = R. Gaus
]
{Mousse au Chocolat}
    
    %\graph
    %{% Bilder
    %    small=pic/glass,    % kleines Bild
    %    big=pic/ingredients % großes (längeres) Bild
    %}
    
    \ingredients
    )\\
        3 & Eier\\
        \unit[200]{ml} & Sahne\\
        \unit[40]{g} & Zucker\\
        \unit[50]{g} & Butter
    }
    
    \preparation
    { % Zubereitung
        \step Eier trennen, Eiweiß und Sahne separat steif schlagen. Butter und Schokolade vorsichtig im Wasserbad schmelzen.
        \step Eigelb in einer großen Schüssel mit \unit[2]{EL} heißem Wasser cremig schlagen, den Zucker einrühren bis die Masse hell und cremig ist.
        \step Die geschmolzene Schokolade unterheben, anschließend sofort Eischnee und Sahne unterheben (nicht mit dem Elektro-Mixer!)
        \step Mindestens 2 Stunden im Kühlschrank kalt stellen. Aber nicht zu kalt servieren.
    }
    
    \hint
    {% Tipp
        Der Schokoladenanteil kann auch gesenkt werden.
    }

\end{recipe}

% Complete recipe example
\begin{recipe}
[% 
    preparationtime = {\unit[1]{h}},
    bakingtime={\unit[1]{h}},
    bakingtemperature={\protect\bakingtemperature{
        fanoven=\unit[230]{\textcelcius},
        topbottomheat=\unit[195]{°C},
        topheat=\unit[195]{°C},
        gasstove=Level 2}},
    portion = {\portion{5-6}},
    %calory={\unit[3]{kJ}},
    source = {Somebody you used know}
]
{Test Recipe}
    
    %\graph
    %{% pictures
    %    small=pic/glass,     % small picture
    %    big=pic/ingredients  % big picture
    %}
    
    \introduction{%
        \lipsum[1]
    }
    
    \ingredients)\\
        3 & Eggs\\
        \unit[200]{ml} & Cream\\
        40 g & Sugar\\
        50 g & Butter
    }
    
    \preparation{%
        \step \blindtext
        \step \blindtext
        \step \blindtext
    }
    
    \suggestion[Variante]
    {%
        \blindtext
    }
    
    \suggestion{%
        \blindtext
    }
    
    \hint{%
        Enjoy typesetting recipes with {\textbf{\Large\LaTeX}} and {\textbf{\Large xcookybooky!}}
    }
    
\end{recipe}

%\include{tex/Complete_Recipe}

\end{document} 